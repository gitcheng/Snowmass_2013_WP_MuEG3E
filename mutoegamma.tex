\section{$\mu^+\to e^+\gamma$}

Recent MEG measurement at PSI~\cite{MEG2013} sets a limit of 
${\cal B}(\mu^+\to e^+\gamma)< 5.7\times 10^{-13}$ at 90\% confidence
level using $3.6\times 10^{14}$ stopped muons on target. The MEG detector
consists of a set of drift chambers and scintillation timing counters,
located inside a superconducting solenoid, and a liquid Xenon 
calorimeter with UV-sensitive photomultiplier tubes, located outside the
solenoid. 

There are two main sources of background. Over 90\% of the background in
the signal region comes from accidental background, that is, a positron
from a regular Michel muon decay combined with a photon from a radiative
muon decay  (RMD) $\mu^+\to e^+ \nu_e\bar\nu_\mu \gamma$.
Most of the remainig background is due to RMD where the neutrinos carry away minimum
energy. The accidental background rate depends on the instantaneous stopping
muon rate $R_\mu$, total integrating data acquisition time $T$, and 
detector resolutions:
\begin{equation}
N_{\rm acc} \propto R_\mu^2 \times \Delta E_\gamma^2 \times\Delta P_e \times
\Delta \Theta_{e\gamma}^2 \times \Delta t_{e \gamma} \times T, 
\end{equation}
where $\Delta E_\gamma$ and $\Delta P_e$ are the resolutions of photon energy
and positron momentum, respectively; $\Delta \Theta_{e\gamma}$ and
$\Delta t_{e \gamma}$ are the resolutions of $e\gamma$ opening angle and
timing.

The MEG Collaboration has proposed an upgrade~\cite{MEGupgrade} aiming to 
improve the sensitivity to $\mu\to e\gamma$ decay 
by one order of magnitude below the current limit, {\it i.e.,} to set a limit at
$\sim 6\times 10^{-14}$ in the absence of signal. They will replace 
their tracker with a lower-mass, higher-granularity device, reduce target
thickness, use a faster timing counter array, and increase the 
granularity of the liquid xenon detector by replacing the PMTs with 
a larger number of smaller solid state photosensors. The sensitivity
estimate is based on a muon stopping rate of $7\times 10^7$ muons/s for 
a three year run, assuming 180 DAQ days per year.

To improve the experimental reach beyond that of the MEG upgrade, one needs to further improve the detector 
sensitivity. 
The photon energy resolution is a major limiting factor in this search.
A pair spectrometer that measures $e^+e^-$ pair tracks from photon
conversions in a thin dense material can greatly improve the photon energy
resolution. This approach was discussed at  2012 Project X Summer Study~\cite{Fritz}. The loss of efficiency due to the small photon conversion 
probability can be compensated for by improved fiducial solid angle coverage and by the higher beam power at Project X at 
Fermilab. 

We have conducted an initial study of this concept using a fast simulation
tool (FastSim) originally developed for the Super$B$ experiment~\cite{SuperB}
using the \babar\
software framework and analysis tools. FastSim allows us to model detector 
components as two-dimensional shells of simple geometries such as cylinders,
cones, disks, and planes. The effect of physical thickness is modeled
parameterically. Coulomb scattering and ionization energy loss are modeled
with the standard parameterization in terms of radiation length and particle
momentum and velocity. Bremsstrahlung and pair production are modeled by
simplified cross-sections. Tracking measurements are described in terms of
the single-hit and two-hit resolution, and the efficiency. Silicon strip
detectors are modeled as two independent orthogonal projections. 
FastSim reconstructs high-level detector objects from simulated hits and
energy deposits using the simulation truth to associate detector objeccts,
bypassing pattern recognition. Errors associated with pattern recognition are
introduced by perturbing the truth-based association, using models based on
\babar\ pattern recognition algorithm performance. The final set of hits on
associated with a track is passed to the \babar\ Kalman filter track fitting
algorithm to obtain reconstructed track parameters.

The FastSim model in this study consists of a thin aluminum stopping target 
and a six-layer
cylindrical silicon detector. A 0.56 mm thick lead (10\% $X_0$) half cylinder 
covering 0--$\pi$ in azimuthal angle at $R = 80$ mm serves as the photon converter.
The target consists of two cones connected at their base; each cone is 50 mm 
long, 5 mm in radius, and 50 $\mu$m thick. Two silicon detector cylinders are
placed close the target for better vertexing resolution; two layers are placed
just outside the Pb converter, and two layers a few cm away. The layout is shown
in Fig.\ref{fig:detscheme}; a signal event display is shown in 
Fig.~\ref{fig:evtdisplay}. The silicon detector is modeled after Super$B$ 
inner silicon striplet modules but thinner. Each layer is formed of 50 $\mu$m thick 
double-sided striplets silicon sensors mounted on 50 $\mu$m of kapton. 
The hit spatial resolution is modeled as a sum of two components with resolutions
of 8 $\mu$m and 20 $\mu$m, and a hit efficiency of 90\%.
The entire detector is placed in a 1T solenoidal
magnetic field.


\begin{figure}[htbp]
\centering
\begin{minipage}[c]{0.47\textwidth}
\centering
\includegraphics[width=\textwidth]{Figures/muegamma-schematic.pdf}
\caption{Schematic drawing (in the plane transverse to the muon beam axis) of the $\mu\to e\gamma$ detector.}
\label{fig:detscheme}
\end{minipage}
\quad
\begin{minipage}[c]{0.47\textwidth}
\centering
\includegraphics[width=\textwidth]{Figures/event_display.pdf}
\caption{FastSim signal event display}
\label{fig:evtdisplay}
\end{minipage}
\end{figure}

We generate muons at rest and have them decay via $\mu^+\to e^+\gamma$
to study the reconstruction efficiency and resolution. 
Approximately 1.3\% of generated signal events are well-reconstructed, 
passing quality and fiducial selection criteria. The photon energy resolution 
is approximately 200~keV (Fig.~\ref{fig:eresol}), similar to the positron momentum
resolution, which 
corresponds to 0.37\% for 52.8 MeV photons. This is a substanial improvement compared 
to the 1.7\%--2.4\% resolution of the current MEG and the 1.0\%--1.1\% resolution 
goal of the MEG upgrade. 

\begin{figure}[ht]
\centering
\includegraphics[width=0.49\textwidth]{Figures/egamma-resol-fit2b.pdf}
\includegraphics[width=0.49\textwidth]{Figures/mumass-resol-fit2b.pdf}
\caption{\label{fig:eresol} Photon energy and $e\gamma$ invariant
mass distributions. Fitted curve is a double-Gaussian distribution.}
\end{figure}


The positron angular resolution is slightly below 10~mrad in both $\theta$ 
and $\phi$ views,
better than current MEG performance but worse than MEG upgrade projection.
The photon direction, determined solely from $e^+e^-$ momenta, has a resolution
similar to that of the positron. It can be further improved  by using
vertex information. Both the  $\gamma\to e^+e^-$ vertex and positron production vertex
(by extrapolating positron track back to the target) have a position resolution
of the order of 100~$\mu$m. Therefore, the photon direction, determined by connecting the two
vertices, has a resolution of the order of 1~mrad (given the lever arm of 80 mm).
As a result, the resolution of the angle between $e^+$ and $\gamma$ is dominated
by $e^+$ angular resolution.

We then use a toy Monte Carlo technique to determine the sensitivity of this
apparatus.
For accidental background, we generate $e^+$ and $\gamma$ from the Michel
spectrum and the RMD spectrum~\cite{Kuno:1999jp}, respectively. Only those momenta
near the end points of the spectra could contribute to the background.
The directions, production points, and production times of $e^+$ and 
$\gamma$ are generated
randomly without correlation. We ignore the other positron originating from the RMD.
For the RMD background, we generate $e^+$ and $\gamma$ according to the theoretical
partial branching fraction formula~\cite{Kuno:1999jp}. Their directions are
correlated, and their production times and positions are identical.
The number of accidental background events is a product of $R_\mu^2$, the partial
branching fractions of the Michel decay and RMD, the selection timing window, the
total DAQ time, phase space factors, and the reconstruction and selection 
efficiencies. For the RMD background, the scaling factor is $R_\mu$, instead of
$R_\mu^2$.

The energies and directions
of the $e^+$ and $\gamma$ are smeared according to the FastSim study
using double-Gaussian functions.
We study the scenarios with timing resolutions of 50~ps and 100~ps. 
The MEG experiment uses 5 independent variables $E_\gamma$, $p_e$, 
$\phi_{e\gamma}$, $\theta_{e\gamma}$, and $\Delta t_{e\gamma}$, to construct
their likelihood function. In our detector, we can take advantage of the excellent
direction resolution of the converted photon. If the photon is produced
at a different point from positron production point, as is the case for accidental backgrounds,
the direction of the $\gamma\to e^+e^-$ momentum and that of the line
connecting the $e^+e^-$ vertex and the primary $e^+$ production point on the target
will be different.  Two additional variables $\Delta\theta_\gamma$ 
and $\Delta\phi_\gamma$ are therefore used in our study. Comparisons between signal
and accidental background are shown in Fig.~\ref{fig:muegamma-vars}.

To estimate the 90\% C.L. upper limit sensitivity, we use a cut-and-count
approach to estimate the background level and then a Feldman-Cousins 
method~\cite{Feldman:1997qc} to calculate the upper limit sensitivity assuming
no signal events are present.



\begin{figure}[htbp]
\includegraphics[width=0.99\textwidth]{Figures/sens-vardists-50ps.pdf}
\caption{Discriminating variables used in the $\mu^+\to e^+\gamma$ search.}
\label{fig:muegamma-vars}
\end{figure}

Figure~\ref{fig:muegamma-sensitivity} shows the background levels, 
signal efficiency, and 90\%
C.L. sensitivity under various selection cuts for 
$R_\mu=1\times 10^{9}$~muons/s, and 50-ps resolution on $t_{e\gamma}$.
A sensitivity of $B(\mu^+\to e^+\gamma)<1.6\times 10^{-14}$ could be reached
with an integrated DAQ time of 1.5 years.
The sensitivity reach
as a function of integrated DAQ time for both 50-ps and 100-ps timing
resolutions is also shown.

Increasing the muon rate further could improve the sensitivity. However,
the sensitivity quickly moves away from the ${\cal O}(1)$ background regime, because the accidental
background grows as $\sim R_\mu^2$. A better approach is to increase the
efficiency and reduce the muon rate to keep the background level low. 
Figure~\ref{fig:muegamma-sens-5x} 
shows a scenario in which the signal efficiency is 5-times higher
and the muon stopping rate is slightly reduced to $R_\mu=7\times 10^{8}$. 
In this scenario, one can reach a sensitivity of $B(\mu^+\to e^+\gamma)<6\times 10^{-15}$.
Such an approach can be realized with multiple layers of thin photon converters and associated silicon tracking layers. Studies of the sensitivity of a multi-converter design are underway.

An alternative version of the photon conversion approach to a $\mu \to e \gamma$ 
experiment has also been discussed~\cite{franco}. In this version, consider a large volume
solendoidal magnet, such as the KLOE coil, which has a radius of 2.9~m, run at a
field of perhaps 0.25~T. A large volume, low mass cylindrical drift chamber 
provides many ($\ge$100) layers of tracking, utilizing small cells and having 
a total number of sense wires approaching $10^5$. Interspersed every ten layers 
is a 0.5 mm W converter shell. There are a sufficient number of points on the 
$e^+$ and $e^-$ tracks from converted photons behind each converter to reach a
 total conversion efficiency of perhaps 80\%, with excellent photon mass 
resolution. 



\begin{figure}[htbp]
   \centering
   \includegraphics[width=0.48\textwidth]{Figures/muegamma-sens-1e9-1p5y-50ps.pdf} 
   \includegraphics[width=0.48\textwidth]{Figures/muegamma-sens-years-1e9-2.pdf} 
   \caption{Left: $B(\mu^+\to e^+\gamma)$ sensitivity optimzation for a given
scenario (see text). Right: sensitivity as a function of integrated DAQ time for both 50-ps and 100-ps $t_{e\gamma}$ resolutions.}
   \label{fig:muegamma-sensitivity}
\end{figure}

\begin{figure}[htbp]
\centering
\includegraphics[width=0.48\textwidth]{Figures/muegamma-sens-7e8-1p5y-50ps-x5sigeff.pdf}
\caption{Left: $B(\mu^+\to e^+\gamma)$ sensitivity optimzation with 5-times
higher signal sensitivity and lower $R_\mu$ than that in 
Fig.~\ref{fig:muegamma-sensitivity}. Best sensitivity is $6\times 10^{-15}$.}
\label{fig:muegamma-sens-5x}
\end{figure}


In summary, using a converted photon to increase the $\mu^+\to e^+\gamma$ detection
sensitivity by improving the photon energy resolution appears to be a promising approach. More detailed studies are 
needed to quantify the requirements in detail, with the goal of improving upon the MEG upgrade
sensitivity by an about order of magnitude.
