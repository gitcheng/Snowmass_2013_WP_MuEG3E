\documentclass[12pt]{article}
\usepackage{epsfig}
\usepackage{subfigure}
\usepackage{rotating}
\usepackage{afterpage}
\usepackage{amsmath} 
\usepackage{amssymb}
\usepackage{relsize}
\usepackage{array}
\usepackage{bm}       %    enables bold math symbols  e.g.  \bm{\gamma}
\usepackage{graphicx}
%%\usepackage{hyperref}      % hypertext links %%ARXIV
%\usepackage{draftwatermark}
%\SetWatermarkScale{5}
%\SetWatermarkLightness{0.9}


\textwidth 6.5in      % 1 in = 2.54 cm = 72.27 pt; 1 cm = 28.45 pt
\textheight 9.0in     % Top to end of text; page number not included
\oddsidemargin 0.0in 
%\evensidemargin 0.5in 
\topmargin -0.5in 
\topskip 0pt \footskip 20pt
\renewcommand\textfraction{0.0}


\input workshopsymbols.tex      %   standard macros for common HEP terms

\def\babar{\mbox{\sl B\hspace{-0.4em} {\small\sl A}\hspace{-0.37em} \sl B\hspace{-0.4em} {\small\sl A\hspace{-0.02em}R}}}


\begin{document}

%\title{Next generation $\mu\to e\gamma$ and $\mu\to eee$ detectors}
\hfill CALT 68-2861
\bigskip
%\title{The next generation of 
%CLFV experiments:\break
%$\mu \to e \gamma$ and  $\mu \to 3e$ at Project X}

%\author{C.-h. Cheng, B. Echenard and D.G. Hitlin}

%\maketitle
\begin{center}
{\Large\bf The next generation of 
CLFV experiments:\break
$\mu \to e \gamma$ and  $\mu \to 3e$ at Project X\break\\}
%\smallskip
{\large C.-h. Cheng, B. Echenard and D.G. Hitlin\break\\}
%\smallskip
\vskip -12pt
{\large California Institute of Technology, Pasadena, California 91125, USA}
\end{center}
\smallskip
\begin{abstract}
We explore the possibilites for extending the sensitivity of current searches for the charged lepton flavor violating decays $\mu\to e \gamma$ and  $\mu\to eee$.  A future facility such as Project X at 
Fermilab could provide a much more intense beam, allowing more sensitive searches, but more
sensitive detectors will be required as well. Current searches are limited by accidental and physics backgrounds, and by the total number of stopped muons. One of the limiting factors in current detectors
for $\mu \to e \gamma$ searches is the photon energy resolution of the calorimeter. We present a study of a
conceptual design of a new detector, using a fast simulation software, that
detects converted $e^+e^-$ pairs from signal photons, taking advantage of the  improved
energy resolution of a pair spectrometer based on a silicon charged particle tracker.  We also study
a related detector design for a next generation $\mu\to eee$ search experiment.
\end{abstract}

\input introduction.tex

\input mutoegamma.tex

\input mutoeee.tex

\input conclusion.tex

\input bibliography.tex



\end{document}
