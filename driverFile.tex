\documentclass[12pt]{article}
\usepackage{epsfig}
\usepackage{subfigure}
\usepackage{rotating}
\usepackage{afterpage}
\usepackage{amsmath} 
\usepackage{amssymb}
\usepackage{relsize}
\usepackage{array}
\usepackage{bm}       %    enables bold math symbols  e.g.  \bm{\gamma}
\usepackage{graphicx}
\usepackage{hyperref}      % hypertext links %%ARXIV


\textwidth 6.5in      % 1 in = 2.54 cm = 72.27 pt; 1 cm = 28.45 pt
\textheight 9.0in     % Top to end of text; page number not included
\oddsidemargin 0.0in 
%\evensidemargin 0.5in 
\topmargin -0.5in 
\topskip 0pt \footskip 20pt
\renewcommand\textfraction{0.0}


\input workshopsymbols.tex      %   standard macros for common HEP terms

\def\babar{\mbox{\sl B\hspace{-0.4em} {\small\sl A}\hspace{-0.37em} \sl B\hspace{-0.4em} {\small\sl A\hspace{-0.02em}R}}}


\begin{document}

%\title{Next generation $\mu\to e\gamma$ and $\mu\to eee$ detectors}

\title{Study of next generation of charged lepton flavor violation experiments at Project X}

\author{Authors}

\maketitle

\begin{abstract}
Current search for charged lepton flavor violating decay $\mu\to e \gamma$ is 
limited by the accidental background. Future facility such as Project X at 
Fermilab could provide a much more intensive beam but one needs a more 
sensitive detector as well. One of the limiting factors in current detectors
is the photon energy resolution of the calorimeter. We present a study of a
conceptual design of a new detector, using a fast simulation software, that
detects converted $e^+e^-$ pairs from signal photons, taking advantage of a much
better energy resolution of a charged particle tracking device. We also study
a similar design for $\mu\to eee$ ...
\end{abstract}

\input introduction.tex

\input mutoegamma.tex

\input mutoeee.tex

\input bibliography.tex



\end{document}
